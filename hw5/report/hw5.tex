% MATH 573 HW 5
% LUKE WUKMER

\documentclass[11pt]{article}

% note: some of these are extremely useful and i don't remember why :o
%\usepackage{savetrees} % disable custom geometry stuff if you do this
\usepackage{titling}    % contol over title & stuff
\usepackage{amsmath, amsthm, amssymb, amsfonts}
\usepackage{amsxtra, amscd, geometry, graphicx}
\usepackage{endnotes}
\usepackage{cancel}
\usepackage{wrapfig}    %inline figs
\usepackage{bm} %allows fancy stuff like bold greek in math mode
\usepackage{alltt}
\usepackage{enumerate} %more/easier control over lists, also see enumitem
%\usepackage[all,cmtip]{xypic}
\usepackage{mathrsfs}
\usepackage{listings} % code with syntax highlighting etc
\usepackage{caption}
\usepackage{tabu}     % more customizable tables
%\usepackage{subfigure}
%\usepackage{subcaption}
%\usepackage[pdftex]{hyperref}
%\usepackage[dvips,bookmarks,bookmarksopen,backref,colorlinks,linkcolor={blue},citecolor={blue},urlcolor={blue}](hyperref}


\usepackage{color}

\definecolor{mygreen}{rgb}{0,0.6,0}
\definecolor{mygray}{rgb}{0.5,0.5,0.5}
\definecolor{mymauve}{rgb}{0.58,0,0.82}

\lstset{ %
  backgroundcolor=\color{white},   % choose the background color; you must add \usepackage{color} or \usepackage{xcolor}
  basicstyle=\footnotesize,        % the size of the fonts that are used for the code
  breakatwhitespace=false,         % sets if automatic breaks should only happen at whitespace
  breaklines=true,                 % sets automatic line breaking
  captionpos=b,                    % sets the caption-position to bottom
  commentstyle=\color{mygreen},    % comment style
  deletekeywords={...},            % if you want to delete keywords from the given language
  escapeinside={\%*}{*)},          % if you want to add LaTeX within your code
  extendedchars=true,              % lets you use non-ASCII characters; for 8-bits encodings only, does not work with UTF-8
  frame=single,	                   % adds a frame around the code
  keepspaces=true,                 % keeps spaces in text, useful for keeping indentation of code (possibly needs columns=flexible)
  keywordstyle=\color{blue},       % keyword style
  language=Python,                 % the language of the code
  otherkeywords={*,...},           % if you want to add more keywords to the set
  numbers=left,                    % where to put the line-numbers; possible values are (none, left, right)
  numbersep=5pt,                   % how far the line-numbers are from the code
  numberstyle=\tiny\color{mygray}, % the style that is used for the line-numbers
  rulecolor=\color{black},         % if not set, the frame-color may be changed on line-breaks within not-black text (e.g. comments (green here))
  showspaces=false,                % show spaces everywhere adding particular underscores; it overrides 'showstringspaces'
  showstringspaces=false,          % underline spaces within strings only
  showtabs=false,                  % show tabs within strings adding particular underscores
  stepnumber=2,                    % the step between two line-numbers. If it's 1, each line will be numbered
  stringstyle=\color{mymauve},     % string literal style
  tabsize=2,	                   % sets default tabsize to 2 spaces
  title=\lstname                   % show the filename of files included with \lstinputlisting; also try caption instead of title
}
% change up the fonts (pick one only)
%\usepackage{times}%
\usepackage{helvet}%
%\usepackage{palatino}%
%\usepackage{bookman}%


% These are italic.
\theoremstyle{plain}
\newtheorem{thm}{Theorem}
\newtheorem*{thm*}{Theorem}
\newtheorem{prop}{Proposition}
\newtheorem*{prop*}{Proposition}
\newtheorem{conj}{Conjecture}
\newtheorem*{conj*}{Conjecture}
\newtheorem{lem}{Lemma}
  \makeatletter
  \@addtoreset{lem}{thm}
  \makeatother 
\newtheorem*{lem*}{Lemma}
\newtheorem{cor}{Corollary}
  \makeatletter
  \@addtoreset{cor}{thm}
  \makeatother 
\newtheorem*{cor*}{Corollary}

%\newtheorem{lem}[thm]{Lemma}
%\newtheorem{remark}[thm]{Remark}
%\newtheorem{cor}[thm]{Corollary}
%\newtheorem{prop}[thm]{Proposition}
%\newtheorem{conj}[thm]{Conjecture}

% These are normal (i.e. not italic).
\theoremstyle{definition}
\newtheorem*{ack*}{Acknowledgements}
\newtheorem*{app*}{Application}
\newtheorem*{apps*}{Applications}
\newtheorem{defn}{Definition}
\newtheorem*{defn*}{Definition}
\newtheorem{eg}{Example}
  \makeatletter
  \@addtoreset{eg}{thm}
  \makeatother 
\newtheorem*{eg*}{Example}
\newtheorem*{egs*}{Examples}
\newtheorem{ex}{Exercise}
\newtheorem*{ex*}{Exercise}
\newtheorem*{quest*}{Question}
\newtheorem{rem}{Remark}
\newtheorem*{rem*}{Remark}
\newtheorem{rems}{Remarks}
\newtheorem*{rems*}{Remarks}
\newtheorem{prob}{Problem}
\newtheorem*{prob*}{Problem}
\newtheorem*{soln*}{Solution}
\newtheorem{soln}{Solution}


% New Commands: Common Math Symbols
\providecommand{\R}{\mathbb{R}}%
\providecommand{\N}{\mathbb{N}}%
\providecommand{\Z}{{\mathbb{Z}}}%
\providecommand{\sph}{\mathbb{S}}%
\providecommand{\Q}{\mathbb{Q}}%
\providecommand{\C}{{\mathbb{C}}}%
\providecommand{\F}{\mathbb{F}}%
\providecommand{\quat}{\mathbb{H}}%

% haha, i originally forked this template from one provided by my abstract
% algebra TA (back in 2012 or something). probably don't need most of these,
% huh. 

% New Commands: Operators
\providecommand{\Gal}{\operatorname{Gal}}%
\providecommand{\GL}{\operatorname{GL}}%
\providecommand{\card}{\operatorname{card}}%
\providecommand{\coker}{\operatorname{coker}}%
\providecommand{\id}{\operatorname{id}}%
\providecommand{\im}{\operatorname{im}}%
\providecommand{\diam}{{\rm diam}}%
\providecommand{\aut}{\operatorname{Aut}}%
\providecommand{\inn}{\operatorname{Inn}}%
\providecommand{\out}{{\rm Out}}%
\providecommand{\End}{{\rm End}}%
\providecommand{\rad}{{\rm Rad}}%
\providecommand{\rk}{{\rm rank}}%
\providecommand{\ord}{{\rm ord}}%
\providecommand{\tor}{{\rm Tor}}%
\providecommand{\comp}{{\text{ $\scriptstyle \circ$ }}}%
\providecommand{\cl}[1]{\overline{#1}}%
\providecommand{\tr}{{\sf trace}}%

\renewcommand{\tilde}[1]{\widetilde{#1}}%
\numberwithin{equation}{section}

% i like the squiggly ones more. add as needed

\renewcommand{\Psi}{\varPsi}

\newcommand*\rfrac[2]{{}^{#1}\!/_{#2}}

% a very fancy dot product \ip{f}{g}
\newcommand\ip[2]{ \left\langle {#1} , {#2} \right\rangle }

% "s.t." for math mode
\providecommand{\st}{\text{ s.t. }}

% \norm{f} and such, super useful
\newcommand{\norm}[1]{\left\lVert#1\right\rVert}

% determinant
%\newcommand{\det}[1]{\textsf{det}\left(#1\right)}

% jacobian
\providecommand{\J}{\textsf{J}}

% this makes the spacing between lines of font a little bigger
%\newcommand{\spacing}[1]{\renewcommand{\baselinestretch}{#1}\large\normalsize}
%\spacing{1.2}

\DeclareMathOperator*{\argmin}{arg\,min}
\DeclareMathOperator*{\argmax}{arg\,max}

\newcommand*\mcol[1]{\overset{\big\uparrow}{\underset{\big\downarrow}{#1}}}

% Makes the margin size a little smaller, i gots stuff to say
\geometry{letterpaper,margin=.8in}

% titling stuff (from package titling)
\posttitle{\par\end{center}}
\setlength{\droptitle}{-.5in}
% END PREAMBLE %%%%%%%%%%%%%%%%%%%%%%%%%
%%%%%%%%%%%%%%%%%%%%%%%%%%%%%%%%%%%%%%%%


\begin{document}

\title{Math 573 HW\textsuperscript{\#}5}
\author{Luke Wukmer}
\date{Fall 2015}
\maketitle \thispagestyle{empty} % remove the page number from the first page
\lstset{language=Python}

%%%% PROBLEM 1
\begin{prob}[Production Problem]

An electronics firm manufactures integrated circuits for radios,
televisions, and stereos. For the next month it has available 1500 units of materials and 920
units of labor. The requirements and selling price of one of each of the above products are
given in the following table.

\begin{center}
\begin{tabu}{>{\itshape}c|ccc}
    \hline
    \rowfont{\itshape} &  Units of Material & Units of Labor & Selling Price (\$) \\
    \hline 
    Radio  & 2    & 1   & 8 \\
    TV     & 12   & 8   & 60 \\
    Stereo & 15   & 6   & 45 
\end{tabu}
\end{center}

First determine a production schedule that maximizes income then solve
the linear programming problem with the \textbf{Simplex Method}.
In particular, with the \textbf{Simplex Tableau}.
\end{prob}
\dotfill % . . . . . . . . . . 
\begin{soln*} 
    Our goal is to maximize the profit function $p(x_1, x_2, x_3) = 8 x_1 + 60 x_2 + 45 x_3$ subject to
    \[
         \begin{cases}
             2 x_1 + 12 x_2 + 15 x_3 \le 1500 \\
             x_1 + 8 x_2 + 6 x_3 \le 920 \\
        x_i \ge 0 \end{cases}
    \]

Convert to equalities by adding the slack variables $x_4, x_5$:
    \[
         \begin{cases}
             2 x_1 + 12 x_2 + 15 x_3 + x_4 = 1500 \\
             x_1 + 8 x_2 + 6 x_3 + x_5 = 920 \\
        x_i \ge 0 \end{cases}
    \]
Finally we convert this into a minimization problem by defining the objective function
\[
        C(\bm{x}) := -p(\bm{x}) = - 8 x_1 - 60 x_2 - 45 x_3 - z_0 
    \]
From this we form the simplex tableau
\begin{tabu}{c|c|c|c|c|c|c}
    ~ & $x_1$ & $x_2$ & $x_3$ & $x_4$ & $x_5$ & \textit{constant} \\
    \hline
    $\bm{x_4}$ & 2 & 12 & 15 & 1 & 0 & 1500 \\
    $\bm{x_5}$ & 1 & 8 & 6 & 0 & 1 & 920 \\
    $\bm{c}$   & -8 & -60 & -45 & 0 & 0 & 0
\end{tabu}

From this we see that $\{x_4, x_5\}$ forms a set of basic variables, and that the objective function 
is written only in terms of non-basic variables. Thus, this system is already in canonical form,
and we can begin the simplex method.

Choose smallest negative $c_i \Rightarrow s = 1$
\begin{align*}
    r &= \argmin \left\{ \frac{b_i}{a_{is}} : 1 \le i \le m , a_{is} > 0 \right\}\\
      &= \argmin \left\{ \underbrace{\frac{1500}{2}}_{i=1} ,
                         \underbrace{\frac{920}{1}}_{i=2} \right\} = 1
\end{align*}
So pivot in variable $x_s \to 2x_1$ in row $r=1$; i.e. replace $R_1 \to \frac{1}{2} R_1$ then
$R_2 \to -R_1 + R_2 \,,\, R_3 \to 8R_1 + R_3 $
\begin{center}
\begin{tabu}{c|c|c|c|c|c|c}
    ~ & $x_1$ & $x_2$ & $x_3$ & $x_4$ & $x_5$ & \textit{constant} \\
    \hline
    $\bm{x_1}$ & 1 & 6 & $\frac{15}{2}$ & 1 & 0 & 750 \\
    $\bm{x_5}$ & 0 & 2 & $-\frac{3}{2}$ & -1 & 1 & 170 \\
    $\bm{c}$   & 0 & -12 & 15 & 8 & 0 & 6000
\end{tabu}
\end{center}

Now $c_2 < 0  \Rightarrow s = 2$.

\begin{align*}
    r &= \argmin \left\{ \frac{b_i}{a_{is}} : 1 \le i \le m , a_{is} > 0 \right\}\\
      &= \argmin \left\{ \frac{750}{6} , \frac{170}{2} \right\} = 2
\end{align*}
So pivot at $2x_2$ in row 2 ; i.e. replace $R_2 \to \frac{1}{2} R_2$ then
$R_1 \to -6R_2 + R_1 \,,\, R_3 \to 12R_2 + R_3 $

\begin{center}
\begin{tabu}{c|c|c|c|c|c|c}
    ~ & $x_1$ & $x_2$ & $x_3$ & $x_4$ & $x_5$ & \textit{constant} \\
    \hline
    $\bm{x_1}$ & 1 & 0 & 3 & 4 & -3 & 240 \\
    $\bm{x_2}$ & 0 & 1 & $-\frac{3}{4}$ & -$\frac{1}{2}$ & $\frac{1}{2}$ & 85 \\
    $\bm{c}$   & 0 & 0 & 6 & 2 & 6 & 7020
\end{tabu}
\end{center}
Since all $c_i$ are nonnegative, we have arrived at a feasible solution in terms of the basic variables
$\{x_1,x_2\}$.
\qed
\end{soln*}
\hrulefill % ___________________________________________________________
%\newpage

%%%% PROBLEM 2
\begin{prob}[Simplex Method]
    \dots
\end{prob}

\begin{soln*}
    \dots
\end{soln*}

\hrulefill
\end{document}
