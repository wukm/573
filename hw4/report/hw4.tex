% MATH 573 HW 3
% LUKE WUKMER

\documentclass[10pt]{article}

% note: some of these are extremely useful and i don't remember why :o
%\usepackage{savetrees} % disable custom geometry stuff if you do this
\usepackage{titling}    % contol over title & stuff
\usepackage{amsmath, amsthm, amssymb, amsfonts}
\usepackage{amsxtra, amscd, geometry, graphicx}
\usepackage{endnotes}
\usepackage{cancel}
\usepackage{wrapfig}    %inline figs
\usepackage{bm} %allows fancy stuff like bold greek in math mode
\usepackage{alltt}
\usepackage{enumerate} %more/easier control over lists, also see enumitem
%\usepackage[all,cmtip]{xypic}
\usepackage{mathrsfs}
\usepackage{listings}
\usepackage{caption}
%\usepackage{subfigure}
%\usepackage{subcaption}
%\usepackage[pdftex]{hyperref}
%\usepackage[dvips,bookmarks,bookmarksopen,backref,colorlinks,linkcolor={blue},citecolor={blue},urlcolor={blue}](hyperref}


\usepackage{color}

\definecolor{mygreen}{rgb}{0,0.6,0}
\definecolor{mygray}{rgb}{0.5,0.5,0.5}
\definecolor{mymauve}{rgb}{0.58,0,0.82}

\lstset{ %
  backgroundcolor=\color{white},   % choose the background color; you must add \usepackage{color} or \usepackage{xcolor}
  basicstyle=\footnotesize,        % the size of the fonts that are used for the code
  breakatwhitespace=false,         % sets if automatic breaks should only happen at whitespace
  breaklines=true,                 % sets automatic line breaking
  captionpos=b,                    % sets the caption-position to bottom
  commentstyle=\color{mygreen},    % comment style
  deletekeywords={...},            % if you want to delete keywords from the given language
  escapeinside={\%*}{*)},          % if you want to add LaTeX within your code
  extendedchars=true,              % lets you use non-ASCII characters; for 8-bits encodings only, does not work with UTF-8
  frame=single,	                   % adds a frame around the code
  keepspaces=true,                 % keeps spaces in text, useful for keeping indentation of code (possibly needs columns=flexible)
  keywordstyle=\color{blue},       % keyword style
  language=Python,                 % the language of the code
  otherkeywords={*,...},           % if you want to add more keywords to the set
  numbers=left,                    % where to put the line-numbers; possible values are (none, left, right)
  numbersep=5pt,                   % how far the line-numbers are from the code
  numberstyle=\tiny\color{mygray}, % the style that is used for the line-numbers
  rulecolor=\color{black},         % if not set, the frame-color may be changed on line-breaks within not-black text (e.g. comments (green here))
  showspaces=false,                % show spaces everywhere adding particular underscores; it overrides 'showstringspaces'
  showstringspaces=false,          % underline spaces within strings only
  showtabs=false,                  % show tabs within strings adding particular underscores
  stepnumber=2,                    % the step between two line-numbers. If it's 1, each line will be numbered
  stringstyle=\color{mymauve},     % string literal style
  tabsize=2,	                   % sets default tabsize to 2 spaces
  title=\lstname                   % show the filename of files included with \lstinputlisting; also try caption instead of title
}
% change up the fonts (pick one only)
%\usepackage{times}%
\usepackage{helvet}%
%\usepackage{palatino}%
%\usepackage{bookman}%


% These are italic.
\theoremstyle{plain}
\newtheorem{thm}{Theorem}
\newtheorem*{thm*}{Theorem}
\newtheorem{prop}{Proposition}
\newtheorem*{prop*}{Proposition}
\newtheorem{conj}{Conjecture}
\newtheorem*{conj*}{Conjecture}
\newtheorem{lem}{Lemma}
  \makeatletter
  \@addtoreset{lem}{thm}
  \makeatother 
\newtheorem*{lem*}{Lemma}
\newtheorem{cor}{Corollary}
  \makeatletter
  \@addtoreset{cor}{thm}
  \makeatother 
\newtheorem*{cor*}{Corollary}

%\newtheorem{lem}[thm]{Lemma}
%\newtheorem{remark}[thm]{Remark}
%\newtheorem{cor}[thm]{Corollary}
%\newtheorem{prop}[thm]{Proposition}
%\newtheorem{conj}[thm]{Conjecture}

% These are normal (i.e. not italic).
\theoremstyle{definition}
\newtheorem*{ack*}{Acknowledgements}
\newtheorem*{app*}{Application}
\newtheorem*{apps*}{Applications}
\newtheorem{defn}{Definition}
\newtheorem*{defn*}{Definition}
\newtheorem{eg}{Example}
  \makeatletter
  \@addtoreset{eg}{thm}
  \makeatother 
\newtheorem*{eg*}{Example}
\newtheorem*{egs*}{Examples}
\newtheorem{ex}{Exercise}
\newtheorem*{ex*}{Exercise}
\newtheorem*{quest*}{Question}
\newtheorem{rem}{Remark}
\newtheorem*{rem*}{Remark}
\newtheorem{rems}{Remarks}
\newtheorem*{rems*}{Remarks}
\newtheorem{prob}{Problem}
\newtheorem*{prob*}{Problem}
\newtheorem*{soln*}{Solution}
\newtheorem{soln}{Solution}


% New Commands: Common Math Symbols
\providecommand{\R}{\mathbb{R}}%
\providecommand{\N}{\mathbb{N}}%
\providecommand{\Z}{{\mathbb{Z}}}%
\providecommand{\sph}{\mathbb{S}}%
\providecommand{\Q}{\mathbb{Q}}%
\providecommand{\C}{{\mathbb{C}}}%
\providecommand{\F}{\mathbb{F}}%
\providecommand{\quat}{\mathbb{H}}%

% haha, i originally forked this template from one provided by my abstract
% algebra TA (back in 2012 or something). probably don't need most of these,
% huh. 

% New Commands: Operators
\providecommand{\Gal}{\operatorname{Gal}}%
\providecommand{\GL}{\operatorname{GL}}%
\providecommand{\card}{\operatorname{card}}%
\providecommand{\coker}{\operatorname{coker}}%
\providecommand{\id}{\operatorname{id}}%
\providecommand{\im}{\operatorname{im}}%
\providecommand{\diam}{{\rm diam}}%
\providecommand{\aut}{\operatorname{Aut}}%
\providecommand{\inn}{\operatorname{Inn}}%
\providecommand{\out}{{\rm Out}}%
\providecommand{\End}{{\rm End}}%
\providecommand{\rad}{{\rm Rad}}%
\providecommand{\rk}{{\rm rank}}%
\providecommand{\ord}{{\rm ord}}%
\providecommand{\tor}{{\rm Tor}}%
\providecommand{\comp}{{\text{ $\scriptstyle \circ$ }}}%
\providecommand{\cl}[1]{\overline{#1}}%
\providecommand{\tr}{{\sf trace}}%

\renewcommand{\tilde}[1]{\widetilde{#1}}%
\numberwithin{equation}{section}

% i like the squiggly ones more. add as needed

\renewcommand{\Psi}{\varPsi}

\newcommand*\rfrac[2]{{}^{#1}\!/_{#2}}

% a very fancy dot product \ip{f}{g}
\newcommand\ip[2]{ \left\langle {#1} , {#2} \right\rangle }

% "s.t." for math mode
\providecommand{\st}{\text{ s.t. }}

% \norm{f} and such, super useful
\newcommand{\norm}[1]{\left\lVert#1\right\rVert}

% determinant
%\newcommand{\det}[1]{\textsf{det}\left(#1\right)}

% jacobian
\providecommand{\J}{\textsf{J}}

% this makes the spacing between lines of font a little bigger
%\newcommand{\spacing}[1]{\renewcommand{\baselinestretch}{#1}\large\normalsize}
%\spacing{1.2}

\newcommand*\mcol[1]{\overset{\big\uparrow}{\underset{\big\downarrow}{#1}}}

% Makes the margin size a little smaller, i gots stuff to say
\geometry{letterpaper,margin=.8in}

% titling stuff (from package titling)
\posttitle{\par\end{center}}
\setlength{\droptitle}{-.5in}
% END PREAMBLE %%%%%%%%%%%%%%%%%%%%%%%%%
%%%%%%%%%%%%%%%%%%%%%%%%%%%%%%%%%%%%%%%%


\begin{document}

\title{Math 573 HW\textsuperscript{\#}4}
\author{Luke Wukmer}
\date{Fall 2015}
\maketitle \thispagestyle{empty} % remove the page number from the first page
\lstset{language=Python}

%%%% PROBLEM 1
\begin{prob}
    Let $f$ be the $2\pi$-periodic function determined by the formula
    \[
            f(x) = \left|{x}\right| \quad \text{for} \,   -\pi \leq x \leq \pi
    \]

\begin{enumerate}[(a)]
    \item Find the Fourier series of $f$.
    \item Show that the Fourier series converges absolutely to $f$.
\end{enumerate}
\end{prob}

\begin{soln*} %In the following discussion, we observe that $f(x)$ is an even function on ${[-\pi,\pi]}$.
    \begin{enumerate}[(a)]
        \item 
            We find the Fourier coefficients $c_n$ for the series
            $\hat{f}(\omega) = \sum_{-\infty}^\infty c_n e^{in\omega} $:
            \begin{align*}
                \text{for } n = 0, \quad
                c_0 &= \frac{1}{2\pi} \int_{-\pi}^{\pi} f(\omega) d\omega
                \,=\, \frac{1}{2\pi}\left[\,\frac{\pi}{2} + \frac{\pi}{2}\,\right]
                \,=\, \boxed{\frac{1}{2} = c_0} \\
                \text{and for } n \neq 0, \quad
                c_n &= \frac{1}{2\pi} \int_{-\pi}^{\pi} f(\omega) e^{- i n\omega} d\omega 
                    \,=\, \frac{1}{2\pi}
                        \left[ \int_{0}^{\pi} \omega e^{-in\omega} d\omega
                        + \int_{-\pi}^{0} \left(-\omega\right) e^{-in\omega} d\omega \right] \\
                        &= \frac{1}{2\pi}
                        \left[ \int_{0}^{\pi} \omega e^{-in\omega} d\omega
                        - \int_{-\pi}^{0} \omega e^{-in\omega} d\omega \right] \\
                \text{Via integration by parts, }   \int_a^b \omega e^{-in\omega} d\omega
                        &= \left.-\frac{1}{in} \omega e^{-in\omega}\right|_a^b 
                        - \left(-\frac{1}{in}\right)\int_a^b e^{-in\omega} d\omega \\
                        &= \frac{i}{n}\left[\, \omega e^{-in\omega}\Big|_a^b
                            + \frac{i}{n}e^{-in\omega}\Big|_a^b \, \right]
                            \,=\, \left.\frac{e^{-in\omega}(1+in\omega)}{n^2}\right|_a^b \\
                            \Longrightarrow \quad c_n \,=\, \cdots
                            &= \frac{1}{2\pi}\left[\, \int_0^\pi + \int_{-\pi}^{0} \,\right] \\
                            &= \frac{1}{2\pi n^2}
                                \left[ (-1)^n (1 + i\pi n) - 1 - 1 + (-1)^n (1 - i\pi n)\right] \\
                                &= \frac{1}{2\pi n^2} \left[ (-1)^n \cdot 2 - 2 \right]
                                = \boxed{\frac{{(-1)}^n - 1}{\pi n^2 } = c_n ,\, n\neq 0}
            \end{align*}
        Thus,
        \[
                \hat{f}(\omega) = \sum_{N=-\infty}^{\infty} c_n e^{in\omega}
                \; ,\;\text{where}\quad  c_n =
                \left\{\begin{array}{cr}
                        \rfrac{1}{2}                    & ,\, n = 0 \\[4pt]
                        \frac{{(-1)}^n - 1}{\pi n^2}    & ,\, n\neq 0
                    \end{array}\right.
            \]
                
        \item
            To show the Fourier series is absolutely convergent, we need to show
            $\displaystyle \sum_{N=-\infty}^\infty \left|\,{c_n} \right|$ is finite. 
            Using Bessel's inequality,
            \begin{align*}
                \sum_{-\infty}^{\infty}|c_n|^2
                    \leq \frac{1}{2\pi} \int_{-\pi}^{\pi} | f(\omega) |^2 d\omega
                    =\frac{1}{2\pi} \int_{-\pi}^{\pi} \omega^2 \,d\omega = \cdots = \frac{\pi^2}{3}
                \end{align*}
            Then by theorem, $S_f^N \rightarrow f$. Flesh this out a bit more eh?
    \end{enumerate}
\end{soln*}

\hrulefill

%%%% PROBLEM 2
\begin{prob}
    Prove that for any $N$,
    \[
            \int_{-\pi}^0  D_N(\omega) d\omega
        =   \int_{0}^{\pi}  D_N(\omega) d\omega = \frac{1}{2}
    \]
    where $D_N(\omega)$ is the $N^{\text{th}}$ Dirichlet kernel,
    $\displaystyle D_N(\omega) = \frac{1}{2\pi}\sum_{-N}^N e^{inw}$\,\,.
\end{prob}
\begin{soln*}
    %If  $N = 0$, we have $\displaystyle D_0(\omega) = \frac{1}{2\pi}$.
    Suppose $N\neq 0$. Then, by reorganizing,
    \begin{align*}
        \int_0^\pi D_N(\omega)d\omega &=
        \frac{1}{2\pi} \int_0^\pi \sum_{n=1}^{|N|}\left[
        e^{in\omega} + e^{-in\omega}\right] d\omega + \frac{1}{2\pi}\int_0^\pi e^{i(0)\omega} d\omega \\
        &= \frac{1}{2\pi} \sum_{n=1}^{|N|}\int_0^\pi \left[
        e^{in\omega} + e^{-in\omega}\right] d\omega + \frac{1}{2\pi}\int_0^\pi d\omega \\
        &= \frac{1}{2\pi} \sum_{n=1}^{|N|}
            \left[ \frac{1}{in} e^{in\omega} -\frac{1}{in} e^{-in\omega}\right]_0^\pi
            + \frac{1}{2} \\ 
        &= \frac{1}{2\pi} \sum_{n=1}^{|N|}
            \left[ \frac{1}{in}\left(
            \left(e^{i\pi}\right)^n - \left(e^{i\pi}\right)^{-n} - 1 + 1
        \right)\right] + \frac{1}{2}\\
        &= \frac{1}{2\pi} \sum_{n=1}^{|N|}
        \left[ \frac{1}{in}\left( (-1)^n - (-1)^{-n}\right)\right] + \frac{1}{2} 
        \,=\, 0 + \frac{1}{2} \,=\, \frac{1}{2}
    \end{align*}
    From the above it is clear that if $N=0$, the first term is nonexistent, and thus
    $\int_0^\pi D_0(\omega) d\omega = \frac{1}{2}$ as well.

    Finally, to show $\int_{-\pi}^0  D_N(\omega) d\omega =   \int_{0}^{\pi}  D_N(\omega) d\omega$,
    note that $\int_{-\pi}^{0} D_N(\omega) d\omega = \int_0^\pi D_N(-\omega) d\omega$.
    The first term is simply reordered and still cancels out, whereas the second term is unaffected,
    and we have the result.
    \qed
\end{soln*}

\hrulefill
%%%% PROBLEM 3

\begin{prob}
Show that for any functions $f,g,h$ on $\R$, $f\star (g \star h) = (f\star g) \star h$, where
\[
        f \star g(x) = \int f(x-y) g(y) dy
    \]
\end{prob}

\begin{soln*}
    From lecture, we know convolution is commutative, i.e. $f\star g = g \star f$.
    We apply this result several times in the following:
    \begin{align*}
        (f \star g ) \star h
        &= \int \left(f \star g(x-y)\right) h(y) dy = \int \left(g \star f(x-y)\right) h(y) dy \\
        &= \int \left[ \int g(x-y-\tau) f(\tau) d\tau \right] h(y) dy \\ \tag{\textbf{\maltese}}
        &= \int \int g(x-y-\tau) f(\tau) h(y) d\tau dy \\ 
        &= \int \left[ \int g(x-\tau-y) h(y) dy \right]f(\tau) d\tau  \\
        &= \int g\star h(x-\tau) f(\tau) d\tau = (g \star h) \star f = f \star (g \star h)
    \end{align*}
    Since both convolution integrals exist, we may apply Fubini's theorem in
    (\textbf{\maltese}) to freely change the order of integration.
    \qed
\end{soln*}

\hrulefill

%%%%PROBLEM 4
\begin{prob}[Linear filtering for image denoising]
    Add this.
\end{prob}

\hrulefill
%%%%PROBLEM 5
\begin{prob}[Hybrid images]
    Add this.
\end{prob}
\end{document}
